\documentclass[12pt]{article}

%DEFINIZIONE DEI PACCHETTI GENERICI
\usepackage[a4paper,top=3.5cm,bottom=2.5cm,left=3cm,right=3.5cm]{geometry}
\usepackage{times}
\usepackage{titlesec}
%\usepackage{lipsum}
\usepackage{titletoc}

%DEFINIZIONE DEI PACCHETTI MATEMATICI
\usepackage{amsmath}
\usepackage{amsfonts}
\usepackage{amssymb}
\usepackage{amsthm}

%STILE DELLE PROPOSIZIONI
\theoremstyle{plain}

% Intestazioni in ingelse
%\newtheorem{thm}{Theorem}[section]
%\newtheorem{lem}[thm]{Lemma}
%\newtheorem{prop}[thm]{Proposition}
%\newtheorem{cor}[thm]{Corollary}
%\newtheorem{defn}[thm]{Definition}
%\newtheorem{rmk}[thm]{Remark}
%\newtheorem{ex}[thm]{Example}
%\newtheorem{prob}[thm]{Problem}
%\newtheorem*{notat*}{Notation}

% Intestazioni in italiano
\newtheorem{thm}{Teorema}[section]
\newtheorem{lem}[thm]{Lemma}
\newtheorem{prop}[thm]{Proposizione}
\newtheorem{cor}[thm]{Corollario}
\newtheorem{defn}[thm]{Definizione}
\newtheorem{rmk}[thm]{Osservazione}
\newtheorem{ex}[thm]{Esempio}
\newtheorem{prob}[thm]{Problema}
\newtheorem*{notat*}{Notazione}

%STILE DELLE SEZIONI
\titleformat{\section}
{\normalfont \scshape \centering}{\thesection.}{0,5em}{}
\titlespacing*{\section}{0pt}{50pt}{0.5cm}

\titleformat{\subsection}
{\normalfont \bfseries }{ \normalfont \thesubsection.}{0,5em}{}
\titlespacing*{\subsection}{0pt}{50pt}{0.5cm}

\titlecontents{section}[0cm]{}
{\normalfont \thecontentslabel. \enspace}
{\hspace*{-5.3em}}
{ \hfill \normalfont \contentspage}

\titlecontents{subsection}[0cm]{}
{\normalfont \thecontentslabel. \enspace}
{\hspace*{-5.3em}}
{ \hfill \normalfont \contentspage}

%OPZIONI PER LA BIBLIOGRAFIA
\usepackage[
backend=bibtex,
style=alphabetic,
]{biblatex}
\addbibresource{bibliography.bib}
\AtNextBibliography{\small}


%INIZIO DEL DOCUMENTO
\begin{document}
	
	\fontsize{11pt}{0pt}
	\begin{center}
		\textbf{NOTES ON COMMUTATIVE ALGEBRA}
	\end{center}

    \fontsize{12pt}{0pt}
	\tableofcontents

    %INIZIA A SCRIVERE QUI

    \section{Ring and Ideals}

    \begin{defn}
        Let $A$ be a non empty set. Consider two operation on $A$ 
        \[ + : A \times A \longrightarrow A\]
        \[ \cdot : A \times A \longrightarrow A,\]

        we define the triplets $(A, +, \cdot)$ to be a ring if:
        \begin{enumerate}
            \item[(1)] $(A, +)$ is an abelian group with identity element $0$;
            \item[(2)] $\cdot$ is an associative operation;
            \item[(3)] it holds the distributive property
            \[ a \cdot (b + c) = a \cdot b + a \cdot c\]
            and \[ (b +c ) \cdot a = b \cdot a + c \cdot a;\]
            \item[(4)] $a \cdot 0 = 0 \cdot a = 0.$
        \end{enumerate}
        We define $(A, +, \cdot)$ to be a unitary ring if there exits an alement $1 \in A$ such that
        \[ a \cdot 1 = 1 \cdot a = a,\, \forall a \in A.\]
        We define $(A, +, \cdot)$ to be a commutative ring if, for all $a,b \in A$ it holds
        \[ a \cdot b = b \cdot a.\]
    \end{defn}

    \begin{rmk}
        We do not require the condition $1 \neq 0$. In the case $1=0$ then we have
        \[ 0 = a \cdot 0 = a \cdot 1 = a, \, \forall a \in A, \]
        therefore $A = 0$.
        The only ring in which $1 = 0$ is the zero ring $A = \{ 0\}$.
    \end{rmk}

    \begin{ex}
        \begin{enumerate}
            \item[(1)] $(\mathbb{Z}, +, \cdot)$ is a unitary commutative ring;
            \item[(2)] Mat$_{n\times n}(k)$ is a unitary ring but is not commutative.
        \end{enumerate}
    \end{ex}

    \begin{notat*}
        In the following we do not specify anymore the operations in the ring, we will only specify the name $A$.  With "ring" we will automatically consider a unitary commutative ring unless specified.
    \end{notat*}

    \begin{defn}
        Let $A$ be a ring, then,
        \begin{enumerate}
            \item[(1)] $A$ is an integral domain if, for all $a, b \in A \setminus \{0\}$, $a  b \neq 0$;
            \item[(2)] $A$ is a field if, for all $a \in A$ there exits an element $b \in A$, such that $a b =1$;
            \item[(3)] defining $A ^ \times = \{ a \in A \mid a \text{ is invertible}\}$, then we define $(A^\times, \cdot)$ to be the commutative group of the ring $A$.
        \end{enumerate}
    \end{defn}
        
    \section{Modules}

    \section{Nakayama Lemma}

    \section{An introduction to Homological Algebra}
  
    \section{Tensor Product}

    \section{Localization}

    \section{Noetherian and Artinian rings}

    \section{Bassissatz and Nullstellensatz}

    \section{Artinian rings structure}

    \section{Integer extension of a ring}    

    \section{Noether normalization theorem}

    \section{Basic Algebraic Geometry}
    
	%STAMPA DELLA BIBLIOGRAFIA
	\printbibliography[title=References and Bibliography]
	
\end{document}
