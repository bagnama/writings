\documentclass[12pt]{article}

%DEFINIZIONE DEI PACCHETTI GENERICI
\usepackage[a4paper,top=3.5cm,bottom=2.5cm,left=3cm,right=3cm]{geometry}
\usepackage{times}
\usepackage{titlesec}
%\usepackage{lipsum}
\usepackage{titletoc}

%DEFINIZIONE DEI PACCHETTI MATEMATICI
\usepackage{amsmath}
\usepackage{amsfonts}
\usepackage{amssymb}
\usepackage{amsthm}
\usepackage{stmaryrd}

%STILE DELLE PROPOSIZIONI
\theoremstyle{plain}

% Intestazioni in ingelse
\newtheorem{thm}{Theorem}[section]
\newtheorem{lem}[thm]{Lemma}
\newtheorem{prop}[thm]{Proposition}
\newtheorem{cor}[thm]{Corollary}
\newtheorem{defn}[thm]{Definition}
\newtheorem{rmk}[thm]{Remark}
\newtheorem{ex}[thm]{Example}
\newtheorem{prob}[thm]{Problem}
\newtheorem*{quest*}{Question}
\newtheorem*{notat*}{Notation}

%STILE DELLE SEZIONI
\titleformat{\section}
{\normalfont \scshape \centering}{\thesection.}{0,5em}{}
%\titlespacing*{\section}{0pt}{50pt}{0.5cm}

\titleformat{\subsection}
{\normalfont \bfseries }{\normalfont \thesubsection.}{0,5em}{}
%\titlespacing*{\subsection}{0pt}{50pt}{0.5cm}

\titlecontents{section}[0cm]{}
{\normalfont \thecontentslabel. \enspace}
{\hspace*{-5.3em}}
{ \hfill \normalfont \contentspage}

\titlecontents{subsection}[0cm]{}
{\normalfont \thecontentslabel. \enspace}
{\hspace*{-5.3em}}
{ \hfill \normalfont \contentspage}

%OPZIONI PER LA BIBLIOGRAFIA
\usepackage[
backend=bibtex,
style=alphabetic,
]{biblatex}
\addbibresource{bibliography.bib}
\AtNextBibliography{\small}

% imposto lo stile dell'abstract
\renewcommand{\abstractname}{\normalfont \scshape \centering Abstract}

%INIZIO DEL DOCUMENTO
\begin{document}

    % impostazione del titolo
	\begin{center}
	    \fontsize{12pt}{0pt} % dimensioni del titolo
        \textbf{NOTES ON COMMUTATIVE ALGEBRA} % titolo
	\end{center}

    \begin{abstract}
        These are notes I took during the commutative algebra course at university of Bologna.
    \end{abstract}

    %impostazione dell'indice
    {
    \fontsize{12pt}{0pt} % dimensioni delle voci
    \tableofcontents % print della tavola dei contenuti
    }

    %INIZIA A SCRIVERE QUI
    \fontsize{12pt}{0pt}
    \section{Ring and Ideals}

    \begin{defn}
        Let $A$ be a non empty set. Consider two operation on $A$ 
        \[ + : A \times A \longrightarrow A\]
        \[ \cdot : A \times A \longrightarrow A,\]

        we define the triplets $(A, +, \cdot)$ to be a ring if:
        \begin{enumerate}
            \item[(1)] $(A, +)$ is an abelian group with identity element $0$;
            \item[(2)] $\cdot$ is an associative operation;
            \item[(3)] it holds the distributive property
            \[ a \cdot (b + c) = a \cdot b + a \cdot c\]
            and \[ (b +c ) \cdot a = b \cdot a + c \cdot a;\]
            \item[(4)] $a \cdot 0 = 0 \cdot a = 0.$
        \end{enumerate}
        We define $(A, +, \cdot)$ to be a unitary ring if there exits an alement $1 \in A$ such that
        \[ a \cdot 1 = 1 \cdot a = a,\, \forall a \in A.\]
        We define $(A, +, \cdot)$ to be a commutative ring if, for all $a,b \in A$ it holds
        \[ a \cdot b = b \cdot a.\]
    \end{defn}

    \begin{rmk}
        We do not require the condition $1 \neq 0$. In the case $1=0$ then we have
        \[ 0 = a \cdot 0 = a \cdot 1 = a, \, \forall a \in A, \]
        therefore $A = 0$.
        The only ring in which $1 = 0$ is the zero ring $A = \{ 0\}$.
    \end{rmk}

    \begin{ex}
        \begin{enumerate}
            \item[(1)] $(\mathbb{Z}, +, \cdot)$ is a unitary commutative ring;
            \item[(2)] Mat$_{n\times n}(k)$ is a unitary ring but is not commutative.
        \end{enumerate}
    \end{ex}

    \begin{notat*}
        In the following we do not specify anymore the operations in the ring, we will only specify the name $A$.  With "ring" we will automatically consider a unitary commutative ring unless specified.
    \end{notat*}

    \begin{defn}
        Given a ring $A$, we define the subset $I \subseteq A$ to be an ideal of $A$ if
        \begin{enumerate}
            \item[(1)] $\forall i, j \in I, \, i + j \in I$;
            \item[(2)] $\forall i \in I, \forall a \in A, \, i a \in I$.
        \end{enumerate}
        We define an ideal $I$ to be proper if $I \neq A$.
    \end{defn}

    \begin{lem}
        $I \subset A$ is proper ideal $\iff$ $1 \not \in I$.
    \end{lem}

    \begin{proof}
        If $1 \in I$, then for all $a \in A$ we have $a \cdot 1 \in I$, so $A \subset I$, therefore $I = A$.
        If $I = A$, since $1 \in A$ then $1 \in I$.
    \end{proof}
    
    We introduce now homomorphisms between rings.
    \begin{defn}
        Let $A, B$ two rings. A map $f:A \longrightarrow B$ is an homomorphism of rings if
        \begin{enumerate}
            \item[(1)] $f(a+b) = f(a) + f(b);$
            \item[(2)] $f(ab) = f(a) f(b).$
        \end{enumerate}
    \end{defn}

    From this definition follows that
    \[ f(0_A) = f(a-a) = f(a) - f(a) = 0_B,\]
    but does not follows that $f(1_A) = 1_B.$
    So in addition to the conditions $(1)$ and $(2)$ we require also
    \begin{enumerate}
        \item[(3)] $f(1_A) = 1_B$.
    \end{enumerate}

    \begin{ex}
        We can define a map \[f : \frac{\mathbb{Z}}{2\mathbb{Z}} \longrightarrow \frac{\mathbb{Z}}{6\mathbb{Z}}\] such that
        \begin{enumerate}
            \centering
            \item[] $0 \mapsto 0$
            \item[] $1 \mapsto 3$
        \end{enumerate}
    \end{ex}

    \begin{defn}
        Given $f:A \longrightarrow B$ we define the kernel of $f$ to be
        \[ Ker(f) := \{ x \in A \mid f(x) = 0\}.\]
    \end{defn}

    \begin{lem}
        Given $f : A \longrightarrow B$ an homomorphism of ring, then
        \begin{enumerate}
            \item[(1)] $Ker(f)$ is an ideal of $A$;
            \item[(2)] $Im(f)$ is a subring of $B$.
        \end{enumerate}
    \end{lem}

    \begin{proof}
        \begin{enumerate}
            \item[(1)] Consider $x, y \in Ker(f)$, then
            \[ f(x + y) = f(x) + f(y) = 0 + 0 = 0.\]
            Therefor $x + y \in Ker(f)$.
            \item[(2)] Consider $x, y \in Im(f)$, then there exists $a, b \in A$ such that $f(a) = x$ and $f(b) = y$. Therefor
            \[ x + y = f(a) + f(b) = f(a + b) \in Im(f).\]
        \end{enumerate}
    \end{proof}

    \begin{defn}
         We define $f$ to be injective if $Ker(f) = 0$.\\ We define $f$ to be surjective if $Im(f) = B$.
    \end{defn}
    
    \begin{rmk}
        \begin{enumerate}
            \item[(1)] $Im(f)$ is not an ideal. Consider $i : \mathbb{Z} \longrightarrow \mathbb{Q}$ the inclusion. Then $Im(i) = \mathbb{Z}$ which does not absorb the product in $\mathbb{Q}$.
            \item[(2)] Given $I \subseteq A$ ideal, then
            \[ I \text{ is a subring } \iff 1 \in I \iff I = A.\]
        \end{enumerate}    
    \end{rmk}

    \begin{defn}
        Let $I \subseteq A$ an ideal. We define, for $x, y \in A$, the equivalent relation
        \[ x \equiv y \mod{I} \iff y - x \in I .\]
        The equivalent classes are denoted with $x + I$ for $x \in A$, and we define the quotient ring as 
        \[ \frac{A}{I} := \{ x + I : x \in A \}, \]
        where the sum and product are well defined as
        \[ [x] + [y] = [x + y] \ \text{ and } \ [x] \cdot [y] = [xy].\]
    \end{defn}

    \begin{quest*}
        Are there other possible notion of quotients of ring? No.
    \end{quest*}

    Suppose to start with an equivalent relation $\sim$ on $A$ such that the sum and the product are well defined in the quotient $A/\sim$.
    Define the projection map $\pi : A \longrightarrow A/\sim$ as $\pi(x) = [x]$. Then $\pi$ is a surjective homomorphism of rings, in fact
    \[ \pi(x+y) = [x+y] = [x] + [y]  = \pi(x) + \pi(y),\]
    \[ \pi(x\cdot y) = [x \cdot y] = [x] \cdot [y] = \pi(x) \cdot \pi(y).\]
    Therefore $Ker(\pi)$ is an ideal of $A$ and we have
    \begin{align*}    
    x \sim y &\iff [x] = [y] \iff \pi(x) = \pi(y) \iff \pi(x-y) = 0 \iff \\ 
    &\iff x - y \in Ker(f) \iff x \equiv y \mod{Ker(f)} \iff x \in y + I.
    \end{align*}

    So we can reduce the notion of quotienting by an equivalence relation to the notion of quotienting by an ideal. These two type of quotients are the same.

    \begin{rmk}
        There is another way do define the quotient ring by defining it as a quotient of groups. In fact if $(A, +, \cdot)$ is a ring then $(A,+)$ is a group and if $I \subset A$ is an ideal, then $(I, +)$ is a normal subgroup:
        \[ \forall a \in A, \, a \cdot I = I = I \cdot a.\]
        Therefore is well defined the abelian group $(A/I, +)$ in which we can define the product as
        \[ [a] \cdot [b] := [a \cdot b].\]
        This product is well defined since
        \[ [a+i] [b+j] = [ab + ib + aj + ij] = [ab], \]
        and the identity element is $[1]$.
        So $(A/I, +, \cdot)$ is the quotient ring.
    \end{rmk}

    \begin{quest*}
        What are the ideal of $A/I$?
    \end{quest*}
    Consider a group $G$, and a normal subgroup $H \subset G$. Then we have the following correspondence
    \[ \{\text{subgroups of G containing H}\} \longleftrightarrow \{ \text{subgroups of G/H} \}\]
    \[ H \longmapsto \pi(H)\]
    \[ \pi^{-1}(\bar{H}) \longmapsfrom \bar{H},\]
    where $\pi : G \longrightarrow G/H$ is the quotient map.

    If we consider a ring $A$ and an ideal $I$ we have the correspondence
    \[ \{\text{additive subgroups of A containing I}\} \longleftrightarrow \{ \text{additive subgroups of A/I} \}.\]

    But the maps $\pi$ and $\pi^{-1}$ preserves ideal and subrings, so we have the correspondence
    \begin{equation} \tag{C1} \label{ideals_correspondece}
    \{\text{ideals/subrings of A containing I}\} \longleftrightarrow \{ \text{ideals/subrings of A/I} \}
    \end{equation}
    \[ I \longmapsto \pi(I)\]
    \[ \pi^{-1}(J) \longmapsfrom J.\]
    
    \begin{ex} The ideals of $\mathbb{Z}$ are the ideals $(d) = d\mathbb{Z}$.
    Consider two ideals $(a)$ and $(b)$. Then 
    \[ (a) \subseteq (b) \iff b \mid a. \]
        \begin{enumerate}
            \item[(1)] Consider $A = \mathbb{Z}$ and $I = (6)$.   Then\\
            \begin{center}
            $\begin{matrix}
                \{ \text{ideals of $\mathbb{Z}$ containing (6)} \}&  & \{ \text{ideals of $\mathbb{Z}/(6)$}\} \\ \vspace{0.2cm} \\
                \mathbb{Z} = (1) & \longmapsto & \mathbb{Z}/(6) \\ \vspace{0.2cm} \\
                (2) & \longmapsto & ([2])\\ \vspace{0.2cm} \\
                (3) & \longmapsto & ([3])\\ \vspace{0.2cm} \\
                (6) & \longmapsto & ([0])            
            \end{matrix}$
            \end{center}

            \item[(2)] Ideals of $\mathbb{Z}/(12)$. For the correspondence we need to find the ideals of $\mathbb{Z}$ containing $(12)$, thus the divisors of $12$:
            \[ b \mid 12 \iff b \in \{ 1,2,3,4,6,12\}.\]
            Therefore we have
            \begin{center}
            $\begin{matrix}
                    \{ \text{ideals of $\mathbb{Z}$ containing (12)} \}&  & \{ \text{ideals of $\mathbb{Z}/(12)$}\} \\ \vspace{0.2cm} \\
                     (1) = \mathbb{Z} & \longmapsto & \mathbb{Z}/(12) \\ \vspace{0.2cm} \\
                     (2) & \longmapsto & ([2])\\ \vspace{0.2cm} \\
                     (3) & \longmapsto & ([3])\\ \vspace{0.2cm} \\
                     (4) & \longmapsto & ([4])\\ \vspace{0.2cm} \\
                     (6) & \longmapsto & ([6])\\ \vspace{0.2cm} \\
                     (12) & \longmapsto & ([12]) = ([0])
            \end{matrix}$
            \end{center}
        \end{enumerate}
    \end{ex}
    
    
    \subsection{Fields and Domains}

    \begin{defn}
        Let $A$ be a ring, then
        \begin{enumerate}
            \item[(1)] $A$ is an integral domain if, for all $a, b \in A$, $a  b = 0 \implies a=0 \text{ or } b =0$;
            \item[(2)] $A$ is a field if, for all $a \in A$ there exits an element $b \in A$, such that $a b =1$. We denote $b$ with $a^{-1}$;
            \item[(3)] defining $A ^ \times = \{ a \in A \mid a \text{ is invertible}\}$, then we define $(A^\times, \cdot)$ to be the commutative group of the ring $A$.
        \end{enumerate}
    \end{defn}

    \begin{lem}
        \[ A \text{ is a field} \implies A \text{ is a integer domain.}\]
    \end{lem}

    \begin{proof}
        Consider $a,b \in A$, such that $ab = 0$ and $b \neq 0$. Then
        \[ a = abb^{-1} = (ab)b^{-1} = 0b^{-1} = 0.\]
    \end{proof}
    
    \begin{ex}
        Consider the ring $(\mathbb{Z}, +, \cdot)$. Then the commutative group associated to $\mathbb{Z}$ is $(\{+1, -1\}, \cdot)$.
    \end{ex}

       
    \begin{defn}
        An ideal $I \subseteq A$ is called principal if it is generated by one element
        \[ I = (a).\]
        We define an integral domain $A$ to be a principal ideal domain (PID) if every ideal in $A$ is principal.
    \end{defn}
        
    \begin{ex}
        \begin{enumerate}
            \item[(1)] $\mathbb{Z}$ and $\mathbb{Z}[i]$ are both PID.
            \item[(2)] Let $k$ be a field, then $k[x]$ is a PID.
            \item[(3)] Let $k$ be a field, then define the ring of formal series as
            \[ k[[x]] := \{ a_0 + a_1 x+ a_2x^2 + \dots \mid a_i \in k, \,  \forall i\}\]
            with the same sum and product as the ones in the ring of polynomial.
            Consider now the series 
            \[ 1 - x + x^2 - x^3 + \dots = \sum_{i \ge 0} (-1)^i x^^i\]
            and observe that $(1+x) (1 - x + x^2 - x^3 + \dots) = 1.$
            Now we can prove that
            \[ a(x) \in k[[x]] \text{ is invertible} \iff a_0 \neq 0.\]
            In fact $a(x) = a_0 + a_1 x + a_2 x^2 + \dots = a_o ( 1 + a_0^{-1} a_1 x + a_0^{-1}x_2 x^2 + \dots)$, defining
            \[ A(x) := a_0^{-1} a_1 x + a_0^{-1}x_2 x^2 + \dots, \]
            we have that
            \[ a(x) = a_0(1+A(x)),\]
            so by what we have seen before, the inverse of $a(x)$ is
            \[ a(x)^{-1} = a_0^{-1} (1-A(x) + A(x)^2-\dots)\]
            So the inverse of $a(x)$ exists if and only if $a_0 \neq 0$.
        \end{enumerate}
    \end{ex}

    \begin{defn}
        Let $A$ be a ring and $I \subset A$ an ideal. We define $I$ to be
        \begin{enumerate}
            \item[(1)] prime if $xy \in I \implies x \in I \text{ or } y \in I$;
            \item[(2)] maximal if $\forall J \subset A$ ideal, $ I \subset J \subseteq A \implies I = J$.
        \end{enumerate}
    \end{defn}

    \begin{prop}
        Let $I \subseteq A$ and ideal, then
        \[ \text{$I$ is maximal} \implies  \text{I is prime}\]
    \end{prop}

    \begin{proof}
        Let $I$ be a maximal ideal, consider $x\cdot y \in I$ and suppose that $x \not \in I$ and $y \not \in I$.
        If $x \not \in I$, then $(x) \not \subseteq I$.
        Therefore $I + (a) \supseteq I$ by maximality of $I$ we have that $I + (a) = A$, the same holds for $(b)$, therefore $I + (b) = A$.
        $1 \in A$, so
        \[ 1\cdot 1 = (i + ax) (j + bx) = ij + iby + jax + abxy,\]
        and since $ab \in I$, we have $1 \in I$, therefore $I = A$.
    \end{proof} 

    \begin{lem}
        Let $A$ be a ring, then
        \begin{enumerate}
            \item[(1)] 
                \[ \text{ $A$ is an integral domain} \iff \text{ $(0)$ is a prime ideal } \]
            \item[(2)]
                \[ \text{ $A$ is a field} \iff \text{ $(0)$ is a maximal ideal } \]
        \end{enumerate}
    \end{lem}

    \begin{proof}
        \begin{enumerate}
            \item[(1)]       
                Let $x, y \in A$, if $(0)$ is a prime ideal, then
                \[ xy = 0 \implies xy \in (0) \implies x \in (0) \text{ or } y \in (0) \implies x = 0 \text{ or } y = 0.\]
                If $A$ is an integral domain, then 
                \[ xy \in (0) \implies xy = 0 \implies x = 0 \text{ or } y = 0 \implies x \in (0) \text{ or } y \in (0).\]
            \item[(2)] 
        \end{enumerate}

    \end{proof}
    
    \begin{lem}
        Let $A$ be a ring, then 
        \[ \text{$I \subset A$ is a prime ideal} \iff A/I \text{ is an integral domain}\]
    \end{lem}

    \begin{proof}
        Consider $I \subseteq A$ an ideal. If $I$ is prime then
        \[ 0 = [x][y] = [xy] \implies xy \in I\]
        but $I$ is a prime ideal, so
        \[ x \in I \text{ or } y \in I\]
        therefore
        \[ [x] = 0 \text{ or } [y] = 0.\]
        Viceversa, if $A/I$ is an integral domain, then
        \[ xy \in I \implies [xy] = [x][y] = 0 \implies [x] = 0 \text{ or } [y] = 0 \implies x \in I \text{ or } y \in I. \]
    \end{proof}

    \begin{lem}
        Let $A$ be a ring. Then 
        \[ \text{ $A$ is a field} \iff \text{ the only ideals of $A$ are $\{0\}$ and $A$ }\]
    \end{lem}

    \begin{proof}
        Let $A$ be a ring. Then given $I \subseteq A$ an ideal, we have two possibilities:
        \begin{enumerate}
            \item[(1)] $I = 0$, and we are finished;
            \item[(2)] $I \neq 0$, in that case we can choose $x \in I$ and observe that 
            \[ 1 = x \cdot x^{-1} \in I \]
            and therefore $I = A$.
        \end{enumerate}

        Viceversa, if the only ideals of $A$ are $0$ and $A$, we consider the ideal generated by $x \in A$, $(x)$. Since $(x) \neq 0$ by hypothesis $(x) = A$, but $1 \in A = (x)$.
        Therefore there exists an element $y \in A$ such that $xy = 1$.
    \end{proof}
    
    
    
    
    \section{Modules}

    \section{Nakayama Lemma}

    \section{An introduction to Homological Algebra}
  
    \section{Tensor Product}

    \section{Localization}

    \section{Noetherian and Artinian rings}

    \section{Bassissatz and Nullstellensatz}

    \section{Artinian rings structure}

    \section{Integer extension of a ring}    

    \section{Noether normalization theorem}

    \section{Basic Algebraic Geometry}
    
	%STAMPA DELLA BIBLIOGRAFIA
	\printbibliography[title=References and Bibliography]
	
\end{document}
